\documentclass{article}
\usepackage{amsmath}
\begin{document}

\section{HLLC overview}

The HLLC solver puts the contact wave back into the HLL solver. 

\begin{enumerate}
\item Get wave speeds $S_L$, $S_\star$, $S_R$.
\item Construct $\mathbf{U}_L^\star$ and $\mathbf{U}_R^\star$.
\item Calculate $\mathbf{F}_*^{hllc}$.
\end{enumerate}



\section{Wave speeds}
Wave speeds are obtained from approximate simple Riemann solvers depending on the left-right states. These solvers are the primitive variable RS (PVRS), the two-rarefaction RS (TRRS), and the two-shock RS (TSRS). If the pressure jump at the interface is less than a user specified ratio (typically, $p_{max}/p_{pmin} < 2$) then the flow is smooth and the PVRS is used to estimate $p_\star$ and $u_\star$. If the pressure jump is larger than this ratio, there is likely either a shock or a rarefaction present. If the interface pressure, $p_\star$, given from the PVRS is less than $p_{min}$, then the rarefaction solver, TRRS, is used, else the shock solver, TSRS, is used. 

The estimates for the three approximate solvers for the pressure and velocity are,

\begin{equation}
p_{pvrs} = \frac{1}{2} ( p_L + p_R) - \frac{1}{2} ( u_R - u_L) C
\end{equation}
\begin{equation}
u_{pvrs} = \frac{1}{2} (u_L + u_R) - \frac{1}{2} \frac{ p_R - p_L}{C}
\end{equation}
\begin{equation}
C = \frac{\rho_L + \rho_R}{2} \frac{a_L + a_R}{2}
\end{equation}

\begin{equation}
p_{trrs} = \left[ \frac{a_L+a_R - \frac{\gamma -1}{2} ( u_R - u_L)}{a_L/p_L^z + a_r/p_R^z}\right]^z
\end{equation}
\begin{equation}
u_{trrs} = \frac{P_{LR} u_L/a_L + u_R/a_R + \frac{2(P_{LR}-1)}{\gamma -1}}{P_{LR}/a_L + 1/a_R}
\end{equation}
\begin{equation}
z = \frac{\gamma -1}{2 \gamma} \qquad P_{LR} = \left( \frac{p_L}{p_R} \right)^z
\end{equation}
\begin{equation}
p_{tsrs} = \frac{g_L(p_0) p_L + g_R(p_0) p_R - (u_R - u_L)}{g_L(p_0) + g_r(p_0)}
\end{equation}
\begin{equation}
u_{tsrs} = \frac{1}{2} (u_L + u_R) + \frac{1}{2}\left[ (p_{tsrs}-p_R) g_R(p_0) - (p_{tsrs} - p_L) g_L(p_0) \right]
\end{equation}
\begin{equation}
g_K (p) =\sqrt{\frac{ A_K}{p + B_K}} \qquad p_0 = max(0,p_{pvrs})
\end{equation}
\begin{equation}
A_K =  \frac{2}{\rho_K ( \gamma + 1 ) } \qquad B_K = \left( \frac{\gamma -1}{\gamma + 1} \right) p_K
\end{equation}

The estimates for the interface pressure and velocity are then,

\begin{equation}
p_\star,u_\star = 
\begin{cases}
p_{pvrs},u_{pvrs} & \frac{p_{max}}{p_{min}} < 2 \\
p_{trrs}, u_{trrs} & \frac{p_{max}}{p_{min}} > 2 \, \text{and} \, p_{pvrs} < p_{max} \\ 
p_{tsrs}, u_{tsrs} & \frac{p_{max}}{p_{min}} > 2 \, \text{and} \, p_{pvrs} > p_{max} \\ 
\end{cases}
\end{equation}

Now that we have $p_\star$ and $u_star$ we can calculate the minimum, maxmimum and intermediate wave speeds as,

\begin{equation}
S_L = u_L - a_L q_L
\end{equation}
\begin{equation}
S_L = u_R + a_R q_R
\end{equation}
\begin{equation}
S_\star = \frac{ p_R - p_L + \rho_L u_L (S_L - u_L) - \rho_R u_R (S_R - u_R)}{\rho_L (S_L - u_L) - \rho_R (S_R - u_R)}
\end{equation}
\begin{equation}
q_K = 
\begin{cases}
1 & p_\star \le p_K \\
\sqrt{ 1 + \frac{\gamma +1}{2 \gamma}\left( \frac{p_\star}{p_K} - 1 \right)} & p_\star > p_K
\end{cases}
\end{equation}

\section{Star region}
Now that we have the wave speeds the conservative left and right states in the starred region are,

\begin{equation}
\mathbf{U}_K^\star = \rho_K \left( \frac{ S_K - u_K}{S_K - S_\star} \right) \left[
\begin{matrix}
1 \\ 
S_\star \\
v_K \\
w_K \\
\frac{E_K}{\rho_K} + (S_\star - u_K) \left[ S_\star  + \frac{p_K}{\rho_K(S_K-u_K)} \right]
\end{matrix}
\right]
\end{equation}
Additionally, any passive scalar is advected in the same way as the tangential velocities, i.e
\begin{equation}
(\rho q)_\star^K = \rho_K \left( \frac{S_K - u_K}{S_K - S_\star} \right) q_k
\end{equation}


\section{Final flux}

Finally, the HLLC flux is,

\begin{equation}
\mathbf{F}_{i+1/2}^{hllc} = 
\begin{cases}
\mathbf{F}_L & 0 \le S_L \\
\mathbf{F}_L + S_L(\mathbf{U}_\star^L - \mathbf{U}_L) & S_L \le 0 \le S_\star \\
\mathbf{F}_R + S_R(\mathbf{U}_\star^R - \mathbf{U}_R) & S_\star \le 0 \le S_R \\
\mathbf{F}_R & 0 \ge S_R \\
\end{cases}
\end{equation}


\end{document}