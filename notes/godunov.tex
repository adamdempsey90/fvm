\documentclass{article}
\usepackage{amsmath}
\usepackage[margin=1in]{geometry}
\newcommand{\pderiv}[1]{\frac{\partial }{\partial #1}}
\newcommand{\ppderiv}[2]{\frac{\partial #1}{\partial #2}}
\newcommand{\del}{\mathbf{\nabla}}
\begin{document}

\section{Godunov Method}
We start with the hyperbolic equations of motion written in vector form,
\begin{equation}
    \partial_t \mathbf{U} + \mathbf{\nabla} \cdot \mathbf{F} = \mathbf{S}
\end{equation}
We integrate this over a cell with volume $V$, and use the divergence theorem to get the integral equations of motion,
\begin{equation}
    \frac{d }{d t} \frac{1}{V} \int dV \, \mathbf{U} + \frac{1}{V} \left( \mathbf{F}^+ - \mathbf{F}^- \right)= \frac{1}{V} \int dV \, \mathbf{S} 
\end{equation}
Define the volume averaged quantities as $\bar{\mathbf{U}} = \frac{1}{V} \int dV \, \mathbf{U}$, so that now,
\begin{equation}
    \frac{d }{d t} \bar{\mathbf{U}} + \frac{1}{V} \left( \mathbf{F}^+ - \mathbf{F}^- \right) = \bar{\mathbf{S}} 
\end{equation}
Now integrate in time from $t=0$ to $t=\Delta t$,
\begin{equation}
    \bar{\mathbf{U}}(\Delta t) - \bar{\mathbf{U}} + \frac{1}{V} \int dt \, \left( \mathbf{F}^+ - \mathbf{F}^- \right) = \int dt \, \bar{\mathbf{S}} 
\end{equation}
Up until this point we haven't made any approximations. The trick now is to evaluate the time-averaged boundary fluxes and source terms. We would also like to retain at least second order accuracy in time and space. To do this we'll use a MUSCLE-Hancock scheme with slope limiters and an approximate Riemann solver. 
\section{MUSCLE-Hancock Scheme}
This summary is taken from Toro p.557.
\begin{enumerate}
    \item Set boundary conditions
    \item Set timestep based on CFL condition.
        \begin{equation}
            \Delta t = C_{cfl} \frac{\Delta x}{S_\text{max}}
        \end{equation}
        where $S_\text{max}$ is the maximum wave speed. This is typically the faster of advection, sound speeds, viscous speeds, etc. 
    \item Data reconstruction and boundary extrapolated values.
        Use the primitive equation,
        \begin{equation}
            \partial_t \mathbf{W} + \mathbf{A}(\mathbf{W}) \partial_x \mathbf{W} = 0
        \end{equation}
        To evolve the boundary extrapolated values half a timestep, 
        \begin{eqnarray}
            \mathbf{W}_L &=& \mathbf{W}_i^n + \frac{1}{2} \left[ \mathbf{I} - \frac{\Delta t}{\Delta x} \mathbf{A}(\mathbf{W}_i^n) \right] \Delta _i , \\
            \mathbf{W}_R &=& \mathbf{W}_{i+1}^n - \frac{1}{2} \left[ \mathbf{I} + \frac{\Delta t}{\Delta x} \mathbf{A}(\mathbf{W}_{i+1}^n) \right] \Delta _{i+1} ,
        \end{eqnarray}
        where $\Delta_i$ are the slopes of the primitive variables to be determined below. 
    \item Solution of Riemann problem at each interface. 
        The Riemann problem uses $\mathbf{W}^{L,R}$ to determine $\mathbf{W}_{i+1/2,j}(x/t)$ in the $x$ direction. The interface fluxes are then,
        \begin{equation}
            \mathbf{F}_{i+1/2,j} = \mathbf{F}\left(\mathbf{W}_{i+1/2,j} (0) \right)
            \qquad 
            \mathbf{G}_{i,j+1/2} = \mathbf{G}\left(\mathbf{W}_{i,j+1/2} (0) \right)
        \end{equation}
        IF your cell is moving with some speed $\mathbf{w} = (w_x, w_y)$ (e.g if you have a Lagrangian mesh) then you would evaluate the fluxes at $x/t = w_x$ and $y/t = w_y$ rather than $x/t  = y/t = 0$.
\end{enumerate}

\subsection{Slopes and Slope-Limiters}
The slopes are,
\begin{equation}
    \Delta_i = \frac{1}{2} (1 + w) \Delta_{i-1/2} + \frac{1}{2} ( 1 - w) \Delta_{i+1/2} 
    \qquad
    \Delta_{i+1/2} = \mathbf{U}_{i+1}^n - \mathbf{U}_i^n
\end{equation}
The simplest limiter to use is the MINBEE/SUBERBEE limiter,
\begin{equation}
\Delta_i = 
\begin{cases}
 \text{max}\left[ 0 , \text{min}(\beta \Delta_{i-1/2}, \Delta_{i+1/2}), \text{min}(\Delta_{i-1/2}, \beta \Delta_{i+1/2}) \right],  & \Delta_{i+1/2} > 0 , \\
 \text{min}\left[ 0 , \text{max}(\beta \Delta_{i-1/2}, \Delta_{i+1/2}), \text{max}(\Delta_{i-1/2}, \beta \Delta_{i+1/2}) \right],  & \Delta_{i+1/2} < 0 
\end{cases}
\end{equation}
where $\beta=1,2$ correspond to the MINBEE and SUPERBEE limiters.


\section{HLLC Riemann Solver}

The HLLC solver puts the contact wave back into the HLL solver. 

\begin{enumerate}
\item Get wave speeds $S_L$, $S_\star$, $S_R$.
\item Construct $\mathbf{U}_L^\star$ and $\mathbf{U}_R^\star$.
\item Calculate $\mathbf{F}_*^{hllc}$.
\end{enumerate}



\subsection{Wave speeds}
Wave speeds are obtained from approximate simple Riemann solvers depending on the left-right states. These solvers are the primitive variable RS (PVRS), the two-rarefaction RS (TRRS), and the two-shock RS (TSRS). If the pressure jump at the interface is less than a user specified ratio (typically, $p_{max}/p_{pmin} < 2$) then the flow is smooth and the PVRS is used to estimate $p_\star$ and $u_\star$. If the pressure jump is larger than this ratio, there is likely either a shock or a rarefaction present. If the interface pressure, $p_\star$, given from the PVRS is less than $p_{min}$, then the rarefaction solver, TRRS, is used, else the shock solver, TSRS, is used. 

The estimates for the three approximate solvers for the pressure and velocity are,

\begin{equation}
p_{pvrs} = \frac{1}{2} ( p_L + p_R) - \frac{1}{2} ( u_R - u_L) C
\end{equation}
\begin{equation}
u_{pvrs} = \frac{1}{2} (u_L + u_R) - \frac{1}{2} \frac{p_R - p_L}{C}
\end{equation}
\begin{equation}
C = \frac{\rho_L + \rho_R}{2} \frac{a_L + a_R}{2}
\end{equation}

\begin{equation}
p_{trrs} = \left[ \frac{a_L+a_R - \frac{\gamma -1}{2} ( u_R - u_L)}{a_L/p_L^z + a_r/p_R^z}\right]^z
\end{equation}
\begin{equation}
u_{trrs} = \frac{P_{LR} u_L/a_L + u_R/a_R + \frac{2(P_{LR}-1)}{\gamma -1}}{P_{LR}/a_L + 1/a_R}
\end{equation}
\begin{equation}
z = \frac{\gamma -1}{2 \gamma} \qquad P_{LR} = \left( \frac{p_L}{p_R} \right)^z
\end{equation}
\begin{equation}
p_{tsrs} = \frac{g_L(p_0) p_L + g_R(p_0) p_R - (u_R - u_L)}{g_L(p_0) + g_r(p_0)}
\end{equation}
\begin{equation}
u_{tsrs} = \frac{1}{2} (u_L + u_R) + \frac{1}{2}\left[ (p_{tsrs}-p_R) g_R(p_0) - (p_{tsrs} - p_L) g_L(p_0) \right]
\end{equation}
\begin{equation}
g_K (p) =\sqrt{\frac{A_K}{p + B_K}} \qquad p_0 = max(0,p_{pvrs})
\end{equation}
\begin{equation}
A_K =  \frac{2}{\rho_K ( \gamma + 1 ) } \qquad B_K = \left( \frac{\gamma -1}{\gamma + 1} \right) p_K
\end{equation}

The estimates for the interface pressure and velocity are then,

\begin{equation}
p_\star,u_\star = 
\begin{cases}
p_{pvrs},u_{pvrs} & \frac{p_{max}}{p_{min}} < 2 \\
p_{trrs}, u_{trrs} & \frac{p_{max}}{p_{min}} > 2 \, \text{and} \, p_{pvrs} < p_{max} \\ 
p_{tsrs}, u_{tsrs} & \frac{p_{max}}{p_{min}} > 2 \, \text{and} \, p_{pvrs} > p_{max} \\ 
\end{cases}
\end{equation}

Now that we have $p_\star$ and $u_\star$ we can calculate the minimum, maxmimum and intermediate wave speeds as,

\begin{equation}
S_L = u_L - a_L q_L
\end{equation}
\begin{equation}
S_L = u_R + a_R q_R
\end{equation}
\begin{equation}
S_\star = \frac{ p_R - p_L + \rho_L u_L (S_L - u_L) - \rho_R u_R (S_R - u_R)}{\rho_L (S_L - u_L) - \rho_R (S_R - u_R)}
\end{equation}
\begin{equation}
q_K = 
\begin{cases}
1 & p_\star \le p_K \\
\sqrt{ 1 + \frac{\gamma +1}{2 \gamma}\left( \frac{p_\star}{p_K} - 1 \right)} & p_\star > p_K
\end{cases}
\end{equation}

\subsection{Star region}
Now that we have the wave speeds the conservative left and right states in the starred region are,

\begin{equation}
\mathbf{U}_K^\star = \rho_K \left( \frac{ S_K - u_K}{S_K - S_\star} \right) \left[
\begin{matrix}
1 \\ 
S_\star \\
v_K \\
w_K \\
\frac{E_K}{\rho_K} + (S_\star - u_K) \left[ S_\star  + \frac{p_K}{\rho_K(S_K-u_K)} \right]
\end{matrix}
\right]
\end{equation}
Additionally, any passive scalar is advected in the same way as the tangential velocities, i.e
\begin{equation}
(\rho q)_\star^K = \rho_K \left( \frac{S_K - u_K}{S_K - S_\star} \right) q_k
\end{equation}


\subsection{Final flux}

Finally, the HLLC flux is,

\begin{equation}
\mathbf{F}_{i+1/2}^{hllc} = 
\begin{cases}
\mathbf{F}_L & 0 \le S_L \\
\mathbf{F}_L + S_L(\mathbf{U}_\star^L - \mathbf{U}_L) & S_L \le 0 \le S_\star \\
\mathbf{F}_R + S_R(\mathbf{U}_\star^R - \mathbf{U}_R) & S_\star \le 0 \le S_R \\
\mathbf{F}_R & 0 \ge S_R \\
\end{cases}
\end{equation}

\section{Equations of motion for orthogonal coordinate system}
For an orthogonal coordinate system $(x_i,x_j,x_k)$ with diagonal metric $g_{ij} = h_ i^2 \delta_{ij}$, scale factors $h_i$, coordinate vectors  $\mathbf{e}_i = h_i \hat{\mathbf{e}}_i $, the volume element is $\Delta V = dv \Delta x_1 \Delta x_2 \Delta x_3$ where $dv \equiv h_1 h_2 h_3$, the surface area elements are, $\Delta S_i = ds_i \Delta x_j \Delta x_k$, and where $ds_i \equiv dv / h_i$, where $i,j,k$ are cyclic indices (so no Einstein summation)

The gradient of a scalar, $\Phi$ is,
\begin{equation}
\del \Phi = \frac{1}{h_i} \ppderiv{\Phi}{x_i} \hat{\mathbf{x}}_i  + \frac{1}{h_j} \ppderiv{\Phi}{x_j} \hat{\mathbf{x}}_j + \frac{1}{h_k} \ppderiv{\Phi}{x_k} \hat{\mathbf{x}}_k 
\end{equation}
The Laplacian is,
\begin{equation}
dv \nabla^2 \Phi = \pderiv{x_i}\left( \frac{ds_i}{h_i} \ppderiv{\Phi}{x_i} \right) + \pderiv{x_j}\left( \frac{ds_j}{h_j} \ppderiv{\Phi}{x_j}\right)  + \pderiv{x_k}\left( \frac{ds_k}{h_k}\ppderiv{\Phi}{x_k}\right) 
\end{equation}
The divergence of a vector $\mathbf{v}$ is,
\begin{equation}
d v (\del \cdot \mathbf{v}) = \pderiv{x_i} \left( ds_i v_i \right) +\pderiv{x_j} \left( ds_j v_j \right) +\pderiv{x_k} \left( ds_k v_k \right)  
\end{equation}
The divergence of a vector $\mathbf{v}$ is,
\begin{equation}
ds_i \left(\del \times \mathbf{v} \right) \cdot \hat{\mathbf{x}}_i = \pderiv{x_j} \left( h_k v_k \right) - \pderiv{x_k} \left( h_j v_j \right)
\end{equation}
The divergence of a tensor, $\mathbf{T}$, is, 
\begin{align} 
d v \left(\del \cdot \mathbf{T} \right) \cdot \hat{\mathbf{x}}_i &= \pderiv{x_i} \left( ds_i T_{ii} \right)  +  \pderiv{x_j} \left( ds_j T_{ij} \right) +  \pderiv{x_k} \left( ds_k T_{ik} \right)  \nonumber \\
&+  T_{ij} ds_j  \frac{1}{h_i} \ppderiv{h_i}{x_j}+ T_{ki} ds_k \frac{1}{h_i} \ppderiv{h_i}{x_k} - T_{jj} ds_i \frac{1}{h_j} \ppderiv{h_j}{x_i} - T_{kk} ds_i \frac{1}{h_k} \ppderiv{h_k}{x_i}
\end{align}

We can simplify this further for symmetric tensors, $\mathbf{T} = \mathbf{S}$, and diagonal tensors, $T_{ij} = P \delta_{i,j}$
\begin{align}
dv \left(\del \cdot \mathbf{S} \right)\cdot \hat{\mathbf{x}}_i &= \pderiv{x_i} \left( ds_i S_{ii} \right)  +  \frac{1}{h_i}\pderiv{x_j} \left( h_i ds_j S_{ij} \right) +  \frac{1}{h_i} \pderiv{x_k} \left( h_i ds_k S_{ik} \right) - S_{jj} h_k \ppderiv{h_j}{x_i} - S_{kk} h_j  \ppderiv{h_k}{x_i}  \\
dv \left(\del \cdot \mathbf{P} \right)\cdot \hat{\mathbf{x}}_i  &= \pderiv{x_i} \left( ds_i P \right)   -  P \ppderiv{(ds_i)}{x_i}  
\end{align}
where again the indices $ijk$ are not summed over but instead are cyclic $i \rightarrow j \rightarrow k$.
The point of this form is that if you have a coordinate system where the scale factors only depend on one of the coordinates, then then the non divergence terms for a symmetric tensor will be zero in two of the directions. This is useful for conservation properties. 
The diagonal tensor non-divergence term evaluates to $-P$.



For the Euler equations we have,

\begin{align}
dv \ppderiv{(\rho v_i)}{t} &+ \pderiv{x_i} \left( ds_i (\rho v_i^2 + P) \right) + \frac{1}{h_i} \pderiv{x_j} \left(h_i ds_j  \rho v_i v_j \right) + \frac{1}{h_i} \pderiv{x_k} \left(h_i ds_k \rho v_i v_k \right)  \nonumber \\ 
&- \rho v_j^2  h_k \ppderiv{h_j}{x_i} - \rho v_k^2 h_j \ppderiv{h_k}{x_i} - P \ppderiv{(ds_i)}{x_i} \\ 
dv \ppderiv{\rho}{t} &+ \pderiv{x_i} \left( ds_i \rho v_i \right) +\pderiv{x_j} \left( ds_j \rho v_j \right) +\pderiv{x_k} \left( ds_k \rho v_k \right)  = 0 \\
dv \ppderiv{E}{t} &+ \pderiv{x_i} \left( ds_i (E + P) v_i \right) +\pderiv{x_j} \left( ds_j (E + P)  v_j \right) +\pderiv{x_k} \left( ds_k (E + P) v_k \right)  = 0 \\
\end{align}
where $E = P/(\gamma-1) + \rho v^2/2$.

All fluxes are then weighted by the surface area of the cell's face in the update equation,
\begin{equation}
\frac{d}{dt} \frac{1}{V} \int dV Q  + \frac{1}{V} \left( S^+ F^+ - S^- F^- \right) = \frac{1}{V} \int dV S
\end{equation}

\subsubsection{Cartsian}
In cartesian all scale factors are unity, $h = 1, ds = 1, dv=1$.
\begin{align}
\ppderiv{(\rho v_i)}{t} &+ \pderiv{x_i} \left(\rho v_i^2 + P \right) +  \pderiv{x_j} \left( \rho v_i v_j \right) + \pderiv{x_k} \left( \rho v_i v_k \right)  = 0 \\ 
 \ppderiv{\rho}{t} &+ \pderiv{x_i} \left( \rho v_i \right) +\pderiv{x_j} \left( \rho v_j \right) +\pderiv{x_k} \left( \rho v_k \right)  = 0
 \end{align}
\subsubsection{Cylindrical}
In cylindrical $(r,\phi,z)$, the only non-unity scale factors are $h_\phi = ds_r = ds_z = dv = r$
\begin{align}
r \ppderiv{(\rho v_r)}{t} &+ \pderiv{r} \left( r \rho v_r^2 + r P \right)  + \pderiv{\phi} \left(  \rho v_r v_\phi \right) + \pderiv{z} \left(r  \rho v_r v_z \right)  - \rho v_\phi^2 - P = 0 \\
r \ppderiv{(\rho v_\phi)}{t} &+ \frac{1}{r} \pderiv{r} \left( r^2 \rho v_r v_\phi \right) + \pderiv{\phi} \left(  \rho v_\phi^2 + P \right) + \frac{1}{r} \pderiv{z} \left(r^2 \rho v_\phi v_z \right)  = 0 \\
r \ppderiv{(\rho v_z)}{t} &+ \pderiv{r} \left( r \rho v_r v_z \right) + \pderiv{\phi} \left( \rho v_\phi v_z \right) + \pderiv{z} \left( r \rho v_z^2 + r P \right) = 0 \\
r \ppderiv{\rho}{t} &+ \pderiv{r} \left( r \rho v_r \right) + \pderiv{\phi} \left(\rho v_\phi \right) + \pderiv{z} \left(r  \rho v_z \right)  
\end{align}
\subsubsection{Spherical}
In spherical $(r,\theta,\phi)$, the non-unity scale factors are, $h_\phi =r \sin \theta, h_\theta = r, ds_r =dv= r^2 \sin \theta, ds_\phi = r,$ and $ds_\theta = r \sin\theta$
\begin{align}
r^2 \sin \theta \ppderiv{(\rho v_r)}{t} &+ \pderiv{r} \left( r^2 \sin \theta (\rho v_r^2 + P) \right) + \pderiv{\theta} \left( r \sin \theta \rho v_r v_\theta \right) + \pderiv{\phi} \left( r \rho v_r v_\phi \right) \nonumber \\
&- r \rho v_\theta^2 - r \sin \theta \rho v_\phi^2  - 2 P r \sin \theta = 0 \\
r^2 \sin \theta \ppderiv{(\rho v_\theta)}{t} &+ \pderiv{r} \left( r^3 \sin \theta \rho v_r v_\theta \right) + \pderiv{\theta} \left( r \sin \theta ( \rho v_\theta^2 + P) \right) + \frac{1}{r} \pderiv{\phi} \left( r^2 \rho v_\phi v_\theta \right) \nonumber \\ 
&- r \cos \theta \rho v_\phi^2 - P r \cos{\theta} = 0 \\
r^2 \sin \theta \ppderiv{(\rho v_\phi)}{t} &+ \frac{1}{r \sin \theta} \pderiv{r} \left( r^3 \sin^2 \theta \rho v_r v_\phi \right) + \frac{1}{r \sin \theta} \pderiv{\theta} \left( r^2 \sin^2 \theta \rho v_\theta v_\phi \right) + \pderiv{\phi} \left( r \rho v_\phi^2 + r P \right) = 0 \\
r^2 \sin \theta \ppderiv{\rho}{t} &+ \pderiv{r} \left( r^2 \sin \theta \rho v_r \right) + \pderiv{\phi} \left(r \rho v_\phi \right) + \pderiv{\theta} \left(r \sin \theta  \rho v_\theta \right)   = 0
\end{align}







\section{CTU}

\end{document}
